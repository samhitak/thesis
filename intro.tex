According to the Pew Research Center, $\frac{2}{3}$ of American adults reported using social media as their primary or secondary source of news in 2016 \cite{pew}.  Nowadays, we rely on social media to stay up-to-date on politics, beauty, sports, and everything in between.  As users of social media, we want the content we see to be as relevant to our interests as possible.  And on the other side of things, content creators and publishers want us to engage with their media.  Thus, our social media is becoming increasingly personalized, with content recommended to us based on our friends, our follows, and our likes.  However, many of us still scroll through our feeds, not clicking or reading anything, until something catches our eye.  Why did that something catch our eye?  That is precisely what content creators want to know to get us to better engage.

In a paper explaining Netflix's collaborative filtering algorithms and the business purpose they serve, a senior engineer on their data engineering team noted that if a user goes to the search bar instead of watching one of the recommended shows and movies in the main section of the home page, their algorithms have failed \cite{netflix}.  Their goal is to present such a well-curated list that a user has no need to search for a specific title.  Most popular social media sites do not have a search bar in the same way that Netflix or Google does -- users cannot search for specific content, only profiles.  As users of social media, we either click on what we see or scroll past it as we do not have the ability to search for specific articles or videos that we might find interesting.  Thus, it is even more important for content creators that social media recommends to each user content they will engage with.

So, what makes users of social media engage with content?  Is it because the content was shared by a specific user, because there was an interesting title, or because or some other factor?  This study seeks to answer these questions and determine what characteristics of a user are most predictive of whether or not they will engage with a specific piece of content.  To do so, a dataset from Facebook users will be collected and used to model the diffusion of content and to predict content engagement in the network represented by these users.

By the end of this project, the goal is to have found the profile characteristics that are most highly correlated with engagement.  Using this information would allow content creators to increase visibility and also allow users of social media to have more interesting and useful social feeds.